%titolo{Oriente - Nomadi}
%autore{}
%album{}
%tonalita{}
%famiglia{}
%gruppo{}
%momenti{}
%identificatore{oriente_nomadi}
%data_revisione{02_Sep_2019}
%trascrittore{DarioMangoni}
%video{}
\beginsong{Oriente - Nomadi}
\beginverse
{\nolyrics Intro: \[B-]}
\endverse \beginverse  
Sento c\[B-]ome un sap\[D]ore am\[B-]aro
Tra i fasti d\[B-]i questa c\[D]ivilt\[B-]à
E i segn\[A]ali che non decifri\[G]amo mai
Insonnol\[E-]iti come si\[G]amo nei tr\[B-]am
\endverse \beginverse  
Provo sempre nostalgia
Per le conchiglie sparse in riva al mare
E le seguo come grandi impronte
Sul diario dell’umanità \[D]
\endverse \beginverse  
L’es\[D]ilio del pensiero poi
Si cons\[A]uma d\[G]entro ai b\[D]ar
E dentro v\[E-]uote autobiograf\[G]ie intendo
Pr\[D]ive di prot\[A]agon\[G]ista
\endverse \beginverse  
Am\[D]or che gu\[G]ardi verso ori\[D]ente verso il m\[A]are
Qual \[D]è il nome ch\[G]e pronunci pi\[D]ano, prima d\[A]i dormire \rep{2}
\endverse \beginverse  
Preferisco l’analfabetismo
Alle false astrazioni
A questo modo così un pò socratico
Al riparo dalle passioni
\endverse \beginverse
\emph{Ripetere ritornello}
\endverse
\endsong
