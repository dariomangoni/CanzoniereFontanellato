%titolo{La vita è per volare}
%autore{}
%album{}
%tonalita{}
%famiglia{}
%gruppo{}
%momenti{}
%identificatore{la_vita_e_per_volare}
%data_revisione{02_Sep_2019}
%trascrittore{DarioMangoni}
%video{}
\beginsong{La vita è per volare}
\beginverse
Tra l\[D]a scogliera e il m\[F#-]ondo senti\[G]eri non ce n'\[D]è
c'è il mare scuro e f\[F#-]ondo e n\[G]oi tra il mare e il ci\[D]el.
Con \[F#7]ali per pieg\[B-]are il v\[F#7]olo fin laggi\[G]ù
e poi rialzarsi in \[D]alto nell'\[E7]universo\[A] blu.
\endverse
\beginchorus
L\[G]a v\[(La)]ita è p\[D]er volare, p\[G]er \[A]invent\[D]are
\[G] scegli di v\[D]ivere\[E7] da primo att\[G]ore.
\[G]Insi\[(La)]eme p\[D]er provare, p\[G]er n\[A]avig\[D]are
\[G] sopra le n\[D]uvole d\[G]al bl\[D]u n\[A]el bl\[D]u.
\endchorus
\beginverse
Quel pellicano grasso che abita laggiù
mi ha detto: "Qui è uno spasso, io non mi muovo più".
Ma è già un bel po' che giro ed ho il sospetto che
della mia ala a tiro ancor di meglio c'è.
\endverse \beginverse
Sono arrivato in cima a quelle rocce sai
e ho visto che se prima sembravan più che mai
lontane ed irreali le nuvole lassù
col vento tra le ali puoi giungerci anche tu.
\endverse \beginverse
Se un acquazzone fitto ti prende in volo un dì
prova a pensare al giorno in cui splendeva il sol
con l'acqua in ogni piuma che sembri un baccalà
che per un gabbiano non è gran dignità.
\endverse \beginverse
Volare non è facile, ci sono giorni che
diresti: "Amico caro basta che voli te".
Ma un nido non è il cielo ed è pur vero che
il pesce è fresco solo se vai a pescarlo te.
\endverse
\endsong
