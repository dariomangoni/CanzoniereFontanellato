%titolo{Scrolla le ali}
%autore{}
%album{}
%tonalita{}
%famiglia{}
%gruppo{}
%momenti{}
%identificatore{scrolla_le_ali}
%data_revisione{02_Sep_2019}
%trascrittore{DarioMangoni}
%video{}
\beginsong{Scrolla le ali}
\beginchorus
Scrolla le \[D]ali, coccin\[G]ella,
e contr\[D]olla un po' l'ant\[A]enna
scrolla le \[G]ali che si p\[D]arte,
un gran v\[G]olo si far\[A]à
prato, b\[G]osco e poi mont\[D]agna,
trover\[G]emo l'aquil\[A]a. \[D]
\endchorus
\beginverse
Si\[D]amo \[D7]otto coccin\[G]ell\[A]e
\[D]e voli\[D7]amo l\[G]ibere e sor\[A]elle.
Gu\[B-]arda, si\[E]amo s\[G]opra un pr\[A]ato
gli \[C]animali ci h\[G]anno salut\[A]ato.
\endverse \beginverse
La cicala sta nel prato
e ci da un consiglio spensierato:
chi va forte lascia soli gli altri
chi va piano sempre solo volerà.
\endverse \beginverse
Siamo nel bosco più vicino
e parliamo col vecchio porcospino
poi l'inverno comincia ad arrivare
un bel pino ci lascia riparare.
\endverse \beginverse
Che fatica siamo già in montagna
già si sente qualcuno che si lagna
una capra ci invita a continuare
sulla vetta potremo festeggiare.
\endverse \beginverse
Ma che bello il viaggio è completato
qui dal monte vediamo il bosco e il prato
con Arcanda possiamo chiacchierare
avventure vogliamo continuare.
\endverse
\endsong
