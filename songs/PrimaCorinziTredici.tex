\beginsong{Prima Corinzi Tredici}
\beginverse
\[Re]Anche se io conosc\[La]essi e parlassi
la l\[Si-]ingua di ogni creat\[Sol]ura di D\[La4]i\[La]o,
anche se un giorno arrivassi a capire
i misteri e le forze che spingono il mondo.
\endverse \beginverse
Anche se dalla mia bocca venissero
scienza e parole ispirate dal cielo
e possedessi pienezza di fede
da muovere i monti e riempire le valli.
\endverse
\beginchorus
M\[Fa]a non avessi la c\[Do]arità
r\[Re-]isuonerei come un br\[Fadim]onz\[La]o
s\[Si&]e non don\[Do]assi la v\[Fa]ita ogni gi\[Re-]orno
sar\[Sol-]ei come un timpano che v\[Do7]ibra da solo.
S\[Fa]e non avessi la c\[Do]arità
n\[Re-]on servirebbero a n\[Fadim]ull\[La]a
g\[Si&]esti d'am\[Do]ore, sorr\[Fa]isi di p\[Re-]ace
sar\[Sol-]ei come un cembalo che su\[Do7]ona per s\[Fa]é.
\endchorus
\beginverse
La carità è paziente, è benigna
conosce il rispetto, non cerca interesse,
la carità non s'adira del torto
subìto non serba nessuna memoria
\endverse \beginverse
La carità non sopporta ingiustizie
dal falso rifugge, del vero si nutre,
la carità si appassiona di tutto,
di tutto ha speranza, di tutti ha fiducia.
\endverse
\beginchorus
Non avrà fine la carità,
scompariranno i profeti
solo tre doni per noi resteranno:
la fede, l'amore ed ancora speranza.
Ma più importante è la carità,
più forte di ogni sapienza,
ciò che è perfetto verrà, sarà un mondo
di gioia e di pace che ci attenderà.
\endchorus
\endsong
