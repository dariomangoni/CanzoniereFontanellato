%titolo{E la strada si apre}
%autore{}
%album{}
%tonalita{}
%famiglia{}
%gruppo{}
%momenti{}
%identificatore{e_la_strada_si_apre}
%data_revisione{02_Sep_2019}
%trascrittore{DarioMangoni}
%video{}
\beginsong{E la strada si apre}
\beginverse
R\[F#-]aggio che b\[E]uca le n\[A]ubi
ed è gi\[E]à cielo ap\[F#-]erto \[E]\[A]\[E]
\[F#-]acqua che sc\[E]ende dec\[A]isa sc\[E]avando da s\[G]é
l'argine p\[F#-]er la v\[B-]ita.
\endverse \beginverse
\[A] La traiettoria di un v\[E]olo che
\[F#-] sull'orizz\[(Mi)]onte di s\[D]era
t\[B-]utto di qu\[C#-]esta nat\[D]ura ha una str\[E]ada per s\[F#-]é. \[E]\[A]\[E]
\endverse \beginverse
\[F#-]Attimo ch\[E]e segue \[A]attimo
un s\[E]alto nel t\[F#-]empo \[E]\[A]\[E]
p\[F#-]assi di un m\[E]ondo che t\[A]ende oram\[E]ai all'unit\[G]à
che non è pi\[F#-]ù dom\[B-]ani
\endverse \beginchorus
\[A] Usiamo allora queste m\[E]ani
\[F#-] scaviamo a f\[E]ondo nel cu\[D]ore
s\[B-]olo scegli\[C#-]endo l'am\[D]ore il m\[E4]ondo vedr\[E]à 
\endchorus
\beginchorus
Che la strada si \[A]apre p\[E]asso dopo p\[D]asso
\[A]ora \[E] su questa strada noi\[B-].
\[C#-] E si spalanca un ci\[F#-]elo 
un m\[E]ondo che rin\[D]asce si può v\[F#-]ivere
\[B-] per l'unit\[E4]à. \[E]
\endchorus
\beginverse
N\[F#-]ave che s\[E]egue una r\[A]otta
in m\[E]ezzo alle \[F#-]onde \[E]\[A]\[E]
u\[F#-]omo che s'\[E]apre la str\[A]ada in una gi\[E]ungla di id\[G]ee
seguendo s\[F#-]empre il s\[B-]ole,
\endverse \beginverse
\[A] Quando si sente asset\[E]ato
\[F#-] deve raggi\[(Mi)]ungere l'\[D]acqua
s\[B-]abbia che n\[E]ella ris\[A]acca rit\[E]orna al m\[F#-]are. \[E]\[A]\[F#]
(Easy: s\[B-]abbia che n\[C#-]ella ris\[D]acca rit\[E4]orna al m\[E]are.)
\endverse
\endsong
