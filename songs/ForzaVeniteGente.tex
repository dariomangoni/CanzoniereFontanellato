\beginsong{Forza venite gente}
\beginverse
F\[Sol]orza venite g\[Re7]ente che in pi\[Sol]azza si v\[Do]a,
un gr\[Sol]ande spett\[Re]acolo c'\[Re7]è.
Fr\[Sol]ancesco al p\[Re7]adre la r\[Sol]oba rid\[Do]à!
R\[Sol]endimi tutti i s\[La-]ol\[Do]di ch\[Re7]e h\[Sol]ai!
\endverse
\beginverse
\[Sol] Eccoli, i tuoi s\[Re7]oldi, tieni p\[Sol]adre, sono tu\[Re7]oi.
\[Sol] Eccoti la gi\[Re7]ubba di vell\[Sol]uto se la vu\[Re7]oi.
\[Mi-] Non mi serve n\[Si7]ulla, con un s\[Mi-]aio me ne andr\[Si7]ò;
\[Mi-] eccoti le sc\[Si7]arpe, solo i pi\[Mi-]edi mi terr\[Re]ò; \[Re7]
\[Sol] Butto via il pass\[Re7]ato, il nome ch\[Sol]e mi hai dato t\[Re7]u;
\[Sol] nudo come un v\[Re7]erme, non ti d\[Sol]evo niente pi\[Re7]ù.
\[Mi-] Non avrai più c\[Si7]asa, più fam\[Mi-]iglia non avr\[Si7]ai
\[Mi-] Ora avrò solt\[Si7]anto un padre ch\[Mi-]e si chiama D\[Re]io!
\endverse
\beginverse
F\[Sol]orza venite g\[Re7]ente che in pi\[Sol]azza si v\[Do]a,
un gr\[Sol]ande spett\[Re]acolo c'\[Re7]è.
Fr\[Sol]ancesco al p\[Re7]adre la r\[Sol]oba rid\[Do]à!
F\[Sol]iglio degener\[La-]at\[Do]o ch\[Re7]e s\[Sol]ei!
\endverse
\beginverse
\[Sol] Non avrai più c\[Re7]asa, più fam\[Sol]iglia non avr\[Re7]ai;
\[Sol] non sei più chi \[Re7]eri, ma sei qu\[Sol]ello che sar\[Re7]ai.
\[Mi-] Figlio della str\[Si7]ada, vagab\[Mi-]ondo sono i\[Si7]o;
\[Mi-] col destino in t\[Si7]asca ora il m\[Mi-]ondo è tutto m\[Re]io. \[Re7]
\[Sol] Ora sono un u\[Re7]omo perché l\[Sol]ibero sar\[Re7]ò,
\[Sol] ora sono r\[Re7]icco perché ni\[Sol]ente più vorr\[Re7]ò.
\[Mi-] Nella tua bis\[Si7]accia pane, f\[Mi-]ame e poes\[Si7]ia.
\[Mi-] Fiori di sper\[Si7]anza segner\[Mi-]anno la mia v\[Re]ia.
\endverse
\beginverse
F\[Sol]orza venite g\[Re7]ente che in pi\[Sol]azza si v\[Do]a,
un gr\[Sol]ande spett\[Re]acolo c'\[Re7]è.
Fr\[Sol]ancesco ha sc\[Re7]elto la s\[Sol]ua libert\[Do]à!
F\[Sol]iglio degener\[La-]at\[Do]o ch\[Re7]e s\[Sol]ei!
\[Sol]Ora sar\[La-]ai div\[Do]ers\[Re]o d\[Do]a n\[Sol]oi!
\endverse
\endsong
