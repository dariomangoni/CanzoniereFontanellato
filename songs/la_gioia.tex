%titolo{La gioia}
%autore{}
%album{}
%tonalita{}
%famiglia{}
%gruppo{}
%momenti{}
%identificatore{la_gioia}
%data_revisione{02_Sep_2019}
%trascrittore{DarioMangoni}
%video{}
\beginsong{La gioia}
\beginverse
Asc\[C]olta il rumore delle \[Fdim]onde del m\[C]are,
ed il canto notturno dei mille pensi\[G7]eri dell'umanit\[C]à.
E dom\[Fdim]ani rip\[C]osa, dopo il traffico d\[Fdim]i questo gi\[C]orno
che di sera s'incanta davanti al tram\[G]onto che il sole le d\[C]à.
E se v\[Fdim]uoi puoi cant\[C]are,
e cantare che hai v\[Fdim]oglia di d\[C]are,
\[Fdim] e pensare che anc\[C]ora nascosta può es\[G]istere la felicit\[C]à.
\endverse
\beginchorus
Perché lo vu\[G]oi, perché tu pu\[C]oi
riconquist\[F]are un s\[(Sol7)]orr\[C]iso;
e puoi gioc\[G]are e puoi cant\[C]are
perché ti han d\[F]etto bug\[C]ie;
ti han raccont\[G]ato che l'hanno ucc\[C]isa,
che han calpest\[F]ato la gi\[C]oia,
perché la gi\[F]oia, perché la gi\[G]oia,
perché la gi\[F]oia è con t\[G]e.
E mag\[F]ari fosse un \[G]attimo; v\[A-]ivila, ti prego!
E mag\[F]ari a denti str\[G]etti non f\[A-]arla fuggire,
anche imm\[F]ersa nel frastu\[G]ono,
tu f\[A-]alla sentire,
hai bis\[F]ogno di gi\[G]oia come m\[F]e.
{\nolyrics \[C]\[G]\[C]\[F] \[C]\[C]\[G7]\[C]}
\endchorus
\beginverse
Anc\[C]ora, è già tardi ma r\[Fdim]imani anc\[C]ora,
a gustare ancora per poco quest'\[G7]aria scoperta stas\[C]era.
E dom\[Fdim]ani rit\[C]orna,
tra la gente che l\[Fdim]otta e disp\[C]era,
\[Fdim] e vedrai che ancora nascosta può es\[G]istere la felicit\[C]à.
\endverse
\endsong
