%titolo{Perfetta Letizia}
%autore{}
%album{}
%tonalita{}
%famiglia{}
%gruppo{}
%momenti{}
%identificatore{perfetta_letizia}
%data_revisione{02_Sep_2019}
%trascrittore{DarioMangoni}
%video{}
\beginsong{Perfetta Letizia}
\beginverse
Fr\[D]ate Leone,
agn\[F#-]ello del Signore
Per qu\[G]anto possa un frate
Sull'\[A]acqua camminare.
\endverse \beginverse
Sanare gli ammalati
O vincere ogni male.
O far vedere i ciechi
E i morti camminare.
\endverse \beginverse
Frate Leone,
pecorella del Signore
Per quanto possa un santo frate
Parlare ai pesci e agli animali
\endverse \beginverse
E possa ammansire i lupi
e farli amici come cani
Per quanto possa lui svelare,
che cosa ci darà il domani
\endverse
\beginchorus
Tu scrivi che questa non è:
Perfetta letizia
Perfetta letizia
Perfetta letizia
\endchorus
\beginverse
Frate Leone,
agnello del Signore
Per quanto possa un Frate
parlare tanto bene
\endverse \beginverse
Da far capire i sordi,
e convertire i ladri
Per quanto anche all'inferno
Lui possa far Cristiani
\endverse
\beginchorus
Tu scrivi che questa non è:
Perfetta letizia
Perfetta letizia
Perfetta letizia
\endchorus
\beginverse
E se in m\[G]ezzo a frate inverno
Tra n\[B7]eve freddo e vento
\endverse \beginverse
Stas\[E]era arriveremo a casa
E b\[G#-]usseremo giù al portone
Bagn\[A]ati, stanchi ed affamati
Ci sc\[B7]ambieranno per due ladri
\endverse \beginverse
Ci scacceranno come cani
Ci prenderanno a bastonate
E al freddo toccherà aspettare
Con Sora Notte e Sora Fame
\endverse \beginverse
E se sapremo pazientare
Bagnati, stanchi e bastonati
Pensando che così Dio vuole
E il Male trasformarlo in bene
\endverse \beginchorus
Qui scrivi che questa è:
Perfetta letizia
Perfetta letizia
Perfetta letizia
\endchorus \beginverse
Frate Leone questa è...
Frate Leone questa è... 
Frate Leone questa è...
Frate Leone questa è...
Perfetta letizia
Perfetta letizia
Perfetta letizia
\endverse
\endsong
