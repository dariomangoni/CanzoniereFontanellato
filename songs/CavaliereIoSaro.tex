\beginsong{Cavaliere io sarò}
\beginverse
In qu\[La-]esto cast\[Mi-]ello fat\[La-]ato o gr\[Sol]ande Re Art\[Do]ù
i tu\[Re-]oi cavalieri han port\[La-]ato del r\[Sol]egno le virt\[La-]ù.
Nel du\[La-]ello la f\[Mi-]orza e il cor\[La-]aggio ci sp\[Sol]ingeranno gi\[Do]à
ma v\[Re-]incere col sabot\[La-]aggio non d\[Sol]à felicit\[La-]à.
\endverse
\beginchorus
Cavali\[Do]ere io sar\[Sol]ò
anche senza il mio cav\[Do]allo perché s\[Sol]o,
ch\[Re-]e non si può st\[La-]are sed\[Mi-]uti ad aspett\[La-]ar
e cos\[Do]ì cercher\[Sol]ò
un modo molto b\[Do]ello se si pu\[Sol]ò
p\[Re-]er riuscire a don\[La-]are qu\[Mi-]ello che ho nel cu\[La-]or.
\endchorus
\beginverse
Un vaso ti posso creare se argilla mi darai
oppure mattoni impastare e mura ne farei.
e cavalcando nel bosco rumore non farò
il verso del gufo conosco, paura non avrò.
\endverse \beginverse
Il mio prezioso mantello riparo diverrà
se lungo la strada un fratello al freddo resterà.
Sul volto un sorriso sereno per ogni avversità
ai piedi dell'arcobaleno ci si ritroverà.
\endverse
\endsong
